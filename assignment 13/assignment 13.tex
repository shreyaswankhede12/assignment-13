% Inbuilt themes in beamer
\documentclass{beamer}

% Theme choice:
\usetheme{CambridgeUS}

% Title page details: 
\title{Assignment 13\\Probability and Random Variables} 
\author{Shreyas Wankhede}
\date{\today}
\institute{IIT Hyderabad}
\logo{\large \LaTeX{}}


\begin{document}

% Title page frame
\begin{frame}
    \titlepage 
\end{frame}

% Remove logo from the next slides
\logo{}


% Outline frame
\begin{frame}{Outline}
    \tableofcontents
\end{frame}


% Lists frame
\section{Question}
\begin{frame}{Question}
\begin{block}{Question 11.2}
Find the innovations filter of process $x(t)$ if \\
$s(\omega)=\dfrac{\omega^4 + 64}{\omega^4 +10\omega^2 +9}$
\end{block}


\end{frame}


% Blocks frame
\section{Solution}
\begin{frame}{Solution}
The power spectrum S(s) of a regular process can be written as product \\
\begin{align}
L(s)L(-s)\hspace{5mm} S(\omega)=|L(j\omega)|^2\nonumber
\end{align}
where $L(s)$ is an innovations filter of $x(t)$\\
A rational spectrum is the ratio of two polynomials in $\omega^2$ because 
\begin{align}
S(\omega)&=S(-\omega)\nonumber\\\vspace{4mm}
S(\omega)&=\dfrac{A(\omega^2)}{B(\omega^2)}\nonumber
\end{align}

    
\end{frame} 


\begin{frame}{solution}
\begin{block}{}
This shows that if s(i) is a root of S(s) , then -S(i) is also a root of S(s).Further more all roots are either real or complex conjugates.From this it follows that the roots of S(s) are symmetrical with respect to $j\omega$ axis.Hence they can be separated in two groups,one with $Re(s)<0$ and right group with $RE(s)>0$.The minimum phase factor or innovation factor is the ratio of two polynomials formed with left roots.
\end{block}


\end{frame}

\begin{frame}
Now,
\begin{align}
s_x(\omega)&=\dfrac{\omega^4 + 64}{\omega^4 +10\omega^2 +9}\nonumber\\
&=\dfrac{\omega^2+4\omega+8}{\omega^2+4\omega+3}\hspace{4mm}\dfrac{\omega^2-4\omega+8}{\omega^2-4\omega+3}\nonumber
\end{align}
Therefore,\\
\begin{block}{}
$ L(s)=\dfrac{\omega^2+4\omega+8}{\omega^2+4\omega+3}$
\end{block}

\end{frame}

\end{document}